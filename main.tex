% !TEX program = xelatex
\documentclass[12pt,hyperref,a4paper,UTF8]{ctexart}
\usepackage{CityUhomework}

%%------------------正文开始-----------------%%
\begin{document}

%%-----------------封面-------------------%%
\cover\
\thispagestyle{empty}% 首页不显示页码

%%-----------------摘要-------------------%%
\begin{abstract}
    请在此处填写摘要
\end{abstract}


%%-----------------目录-------------------%%
\newpage
\tableofcontents

%%-----------------正文从此处开始-------------------%%
\newpage

\section{模板说明}
默认页边距为2.5cm,中文宋体,英文Times New Roman,字号为12pt。
\subsection{小节}
\subsubsection{小小节}
\section{样例}
\textbf{加粗文本}

\textit{倾斜文本}

\underline{下划线文本}

项目编号:

\begin{itemize}
    \item XXX
    \item XXX
    \item XXX
\end{itemize}

\begin{enumerate}
    \item XXX
    \item XXX
    \item XXX
\end{enumerate}

行内公式:$\int_a^b f(x)dx = F(b)-F(a)$

数学公式排版样例:
\begin{equation}\label{eq:1}
    \int_a^b f(x)dx = F(b)-F(a)
\end{equation}

\begin{equation}\label{eq:2}
    E = mc^2
\end{equation}

\begin{equation}\label{eq:3}
    x^2 \geq 0 \qquad \text{for all } x \in \mathbb{R}
\end{equation}

\begin{equation}\label{eq:4}
    \lim_{n \to \infty}
    \sum_{k=1}^n \frac{1}{k^2}
    = \frac{\pi^2}{6}
\end{equation}

chi-squared distribution:
\begin{equation}\label{eq:5}
    f(y) =
    \begin{cases}
        \frac{1}{2^{k/2}\Gamma(k/2)} x^{k/2-1} e^{-x/2} & y>0              \\
        0                                               & \text{otherwise}
    \end{cases}
\end{equation}

多行公式:
\begin{multline} \label{eq:6}
    a + b + c + d + e + f
    + g + h + i \\
    = j + k + l + m + n\\
    = o + p + q + r + s\\
    = t + u + v + x + z
\end{multline}

\begin{align} \label{eq:7}
    a & = b + c \\
      & = d + e
\end{align}

矩阵:
\begin{equation} \label{eq:8}
    \begin{bmatrix}
        x_{11} & x_{12} & \ldots & x_{1n} \\
        x_{21} & x_{22} & \ldots & x_{2n} \\
        \vdots & \vdots & \ddots & \vdots \\
        x_{n1} & x_{n2} & \ldots & x_{nn} \\
    \end{bmatrix}
\end{equation}

定理:
\newtheorem{mass-energy equivalence}{质能方程}[section]
\begin{mass-energy equivalence} \label{thm:1}
$E = mc^2$
\end{mass-energy equivalence}

插入表格:
\begin{table}[h]
    \begin{tabular}{|c|c|}% 通过添加 | 来表示是否需要绘制竖线
        \hline  % 在表格最上方绘制横线
        (1,1) & (1,2) \\
        \hline  %在第一行和第二行之间绘制横线
        (2,1) & (2,2) \\
        \hline % 在表格最下方绘制横线
    \end{tabular}
\end{table}

插入图片:
[scale=] 中的数可以控制图片大小;后面的括号表示图片的路径,请把图片上传到figures文件夹中;caption表示图片的标题
\begin{figure}[h]
    \centering
    \includegraphics[scale=0.1]{figures/FDS logo.jpg}
    \caption{在此填写图片的标题}
\end{figure}



\end{document}
