% !TEX program = xelatex
\documentclass[12pt,hyperref,a4paper,UTF8]{ctexart}
\usepackage{CityUhomework}
\usepackage{listings}
\usepackage{xcolor}
% 定义可能使用到的颜色
\definecolor{CPPLight}  {HTML} {686868}
\definecolor{CPPSteel}  {HTML} {888888}
\definecolor{CPPDark}   {HTML} {262626}
\definecolor{CPPBlue}   {HTML} {4172A3}
\definecolor{CPPGreen}  {HTML} {487818}
\definecolor{CPPBrown}  {HTML} {A07040}
\definecolor{CPPRed}    {HTML} {AD4D3A}
\definecolor{CPPViolet} {HTML} {7040A0}
\definecolor{CPPGray}  {HTML} {B8B8B8}
\definecolor{keywordcolor}{rgb}{0.8,0.1,0.5}
\definecolor{webgreen}{rgb}{0,.5,0}
\definecolor{bgcolor}{rgb}{0.92,0.92,0.92}

\lstset{
    breaklines = true,                                   % 自动将长的代码行换行排版
    extendedchars=false,                                 % 解决代码跨页时,章节标题,页眉等汉字不显示的问题
    columns=fixed,       
    numbers=left,                                        % 在左侧显示行号
    basicstyle=\zihao{-5}\ttfamily,
    numberstyle=\small,
    frame=none,                                          % 不显示背景边框
    % backgroundcolor=\color[RGB]{245,245,244},            % 设定背景颜色
    keywordstyle=\color[RGB]{40,40,255},                 % 设定关键字颜色
    numberstyle=\footnotesize\color{darkgray},           % 设定行号格式
    commentstyle=\it\color[RGB]{0,96,96},                % 设置代码注释的格式
    stringstyle=\rmfamily\slshape\color[RGB]{128,0,0},   % 设置字符串格式
    showstringspaces=false,                              % 不显示字符串中的空格
    % frame=leftline,topline,rightline, bottomline         %分别对应只在左侧,上方,右侧,下方有竖线
    frame=shadowbox,                                     % 设置阴影
    rulesepcolor=\color{red!20!green!20!blue!20},        % 阴影颜色
    basewidth=0.6em,
}

\lstdefinestyle{CPP}{
    language=c++,                                        % 设置语言
    morekeywords={alignas,continute,friend,register,true,alignof,decltype,goto,
    reinterpret_cast,try,asm,defult,if,return,typedef,auto,delete,inline,short,
    typeid,bool,do,int,signed,typename,break,double,long,sizeof,union,case,
    dynamic_cast,mutable,static,unsigned,catch,else,namespace,static_assert,using,
    char,enum,new,static_cast,virtual,char16_t,char32_t,explict,noexcept,struct,
    void,export,nullptr,switch,volatile,class,extern,operator,template,wchar_t,
    const,false,private,this,while,constexpr,float,protected,thread_local,
    const_cast,for,public,throw,std},
    emph={map,set,multimap,multiset,unordered_map,unordered_set,
    unordered_multiset,unordered_multimap,vector,string,list,deque,
    array,stack,forwared_list,iostream,memory,shared_ptr,unique_ptr,
    random,bitset,ostream,istream,cout,cin,endl,move,default_random_engine,
    uniform_int_distribution,iterator,algorithm,functional,bing,numeric,},
    emphstyle=\color{CPPViolet}, 
}


\lstdefinestyle{Python}{
    language=Python,
}

%%------------------Beginning of the text-----------------%%
\begin{document}

%%-----------------Cover------------------%%
\cover
\thispagestyle{empty}% Home page does not show page numbers
\newpage
%%-----------------Abstract-------------------%%
\begin{abstract}
    Please fill in the abstract here
\end{abstract}


%%-----------------Catalog-------------------%%
\newpage
\tableofcontents

%%-----------------Main text starts here-------------------%%
\newpage

\section{Template Description}
Default margins are 2.5cm, Chinese Song font, English Times New Roman, font size 12pt.
\subsection{bar}
\subsubsection{sub bar}
\section{Example}

\subsection{Insert text}
\textbf{bold text}

\textit{Skewed text}

\underline{underscore text}

\subsection{Insert list}
Item number:

\begin{itemize}
    \item XXX
    \item XXX
    \item XXX
\end{itemize}

\begin{enumerate}
    \item XXX
    \item XXX
    \item XXX
\end{enumerate}

\subsection{Insert mathematical formulas}
Inner equation:$\int_a^b f(x)dx = F(b)-F(a)$

Sample maths formula layout:
\begin{equation}\label{eq:1}
    \int_a^b f(x)dx = F(b)-F(a)
\end{equation}

\begin{equation}\label{eq:2}
    E = mc^2
\end{equation}

\begin{equation}\label{eq:3}
    x^2 \geq 0 \qquad \text{for all } x \in \mathbb{R}
\end{equation}

\begin{equation}\label{eq:4}
    \lim_{n \to \infty}
    \sum_{k=1}^n \frac{1}{k^2}
    = \frac{\pi^2}{6}
\end{equation}

chi-squared distribution:
\begin{equation}\label{eq:5}
    f(y) =
    \begin{cases}
        \frac{1}{2^{k/2}\Gamma(k/2)} x^{k/2-1} e^{-x/2} & y>0              \\
        0                                               & \text{otherwise}
    \end{cases}
\end{equation}

Multi-line formulas:
\begin{multline} \label{eq:6}
    a + b + c + d + e + f
    + g + h + i \\
    = j + k + l + m + n\\
    = o + p + q + r + s\\
    = t + u + v + x + z
\end{multline}

\begin{align} \label{eq:7}
    a & = b + c \\
      & = d + e
\end{align}

Matrix:
\begin{equation} \label{eq:8}
    \begin{bmatrix}
        x_{11} & x_{12} & \ldots & x_{1n} \\
        x_{21} & x_{22} & \ldots & x_{2n} \\
        \vdots & \vdots & \ddots & \vdots \\
        x_{n1} & x_{n2} & \ldots & x_{nn} \\
    \end{bmatrix}
\end{equation}

Theorem:
\newtheorem{mass-energy equivalence}{mass-energy equivalence}[section]
\begin{mass-energy equivalence} \label{thm:1}
$E = mc^2$
\end{mass-energy equivalence}

\subsection{Insert table and picture}
Insert the table:
\begin{table}[h]
    \begin{tabular}{|c|c|}% Indicate whether vertical lines need to be drawn by adding |.
        \hline  % Draw a horizontal line at the top of the table
        (1,1) & (1,2) \\
        \hline  % Drawing a horizontal line between the first and second rows
        (2,1) & (2,2) \\
        \hline % Draw a horizontal line at the bottom of the table
    \end{tabular}
\end{table}

Insert picture:
The number in [scale=] controls the size of the image; the parentheses after it indicate the path of the image, please upload the image to the figures folder; the caption indicates the title of the image.
\begin{figure}[h]
    \centering
    \includegraphics[scale=0.1]{figures/FDS logo.jpg}
    \caption{Fill in the title of the image here}
\end{figure}

\subsection{Insert code}
use\verb|lstlisting| configuration
\begin{lstlisting}[style=CPP, title="c++ code"]
#include <iostream>
using namespace std;
int main() {
    cout << "Hello, World!" << endl;
    return 0;
}

\end{lstlisting}

\begin{lstlisting}[style=Python, title="Python code"]
import numpy as np
import matplotlib.pyplot as plt

x = np.linspace(0, 10, 100)
y = np.sin(x)

plt.plot(x, y)
plt.show()
\end{lstlisting}

\subsection{Insert reference}

Just use \verb|\cite{}|

   \textit{ The literature cited here  \cite{liu2023backdoor}  The literature cited here \cite{du2024sequential} }

%%----------- Reference -------------------%%
% Fill in the references in the reference.bib file, which will be automatically generated here

\reference


\end{document}
