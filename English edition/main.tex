% !TEX program = xelatex
\documentclass[12pt,hyperref,a4paper,UTF8]{article}
\usepackage{CityUhomework}

%%------------------Beginning of the text-----------------%%
\begin{document}

%%-----------------Cover------------------%%
\cover
\thispagestyle{empty}% Home page does not show page numbers
\newpage
%%-----------------Abstract-------------------%%
\begin{abstract}
    Please fill in the abstract here
\end{abstract}


%%-----------------Catalog-------------------%%
\newpage
\tableofcontents

%%-----------------Main text starts here-------------------%%
\newpage

\section{Template Description}
Default margins are 2.5cm, Chinese Song font, English Times New Roman, font size 12pt.
\subsection{bar}
\subsubsection{sub bar}
\section{Example}
\textbf{bold text}

\textit{Skewed text}

\underline{underscore text}

Item number:

\begin{itemize}
    \item XXX
    \item XXX
    \item XXX
\end{itemize}

\begin{enumerate}
    \item XXX
    \item XXX
    \item XXX
\end{enumerate}

Inner equation:$\int_a^b f(x)dx = F(b)-F(a)$

Sample maths formula layout:
\begin{equation}\label{eq:1}
    \int_a^b f(x)dx = F(b)-F(a)
\end{equation}

\begin{equation}\label{eq:2}
    E = mc^2
\end{equation}

\begin{equation}\label{eq:3}
    x^2 \geq 0 \qquad \text{for all } x \in \mathbb{R}
\end{equation}

\begin{equation}\label{eq:4}
    \lim_{n \to \infty}
    \sum_{k=1}^n \frac{1}{k^2}
    = \frac{\pi^2}{6}
\end{equation}

chi-squared distribution:
\begin{equation}\label{eq:5}
    f(y) =
    \begin{cases}
        \frac{1}{2^{k/2}\Gamma(k/2)} x^{k/2-1} e^{-x/2} & y>0              \\
        0                                               & \text{otherwise}
    \end{cases}
\end{equation}

Multi-line formulas:
\begin{multline} \label{eq:6}
    a + b + c + d + e + f
    + g + h + i \\
    = j + k + l + m + n\\
    = o + p + q + r + s\\
    = t + u + v + x + z
\end{multline}

\begin{align} \label{eq:7}
    a & = b + c \\
      & = d + e
\end{align}

Matrix:
\begin{equation} \label{eq:8}
    \begin{bmatrix}
        x_{11} & x_{12} & \ldots & x_{1n} \\
        x_{21} & x_{22} & \ldots & x_{2n} \\
        \vdots & \vdots & \ddots & \vdots \\
        x_{n1} & x_{n2} & \ldots & x_{nn} \\
    \end{bmatrix}
\end{equation}

Theorem:
\newtheorem{mass-energy equivalence}{mass-energy equivalence}[section]
\begin{mass-energy equivalence} \label{thm:1}
$E = mc^2$
\end{mass-energy equivalence}

Insert the table:
\begin{table}[h]
    \begin{tabular}{|c|c|}% Indicate whether vertical lines need to be drawn by adding |.
        \hline  % Draw a horizontal line at the top of the table
        (1,1) & (1,2) \\
        \hline  % Drawing a horizontal line between the first and second rows
        (2,1) & (2,2) \\
        \hline % Draw a horizontal line at the bottom of the table
    \end{tabular}
\end{table}

Insert picture:
The number in [scale=] controls the size of the image; the parentheses after it indicate the path of the image, please upload the image to the figures folder; the caption indicates the title of the image.
\begin{figure}[h]
    \centering
    \includegraphics[scale=0.1]{figures/FDS logo.jpg}
    \caption{Fill in the title of the image here}
\end{figure}



\end{document}
